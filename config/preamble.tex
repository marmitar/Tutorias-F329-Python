\usepackage{graphicx}
\graphicspath{{recursos/}}

% cor das caixas
\usepackage{xcolor}
\definecolor{verde}{RGB}{22,159,37}

% pacote de configuração
\usepackage[
    nome = Python,
    cor  = verde,
    logo = logo.pdf,
    link = https://www.python.org/
]{pacotes/tutorial}

\newcommand{\python}{\software}
\newcommand{\novonome}[2]{
    \newcommand{#1}{%
        \texttt{#2}
    }
}
\novonome{\matplotlib}{Matplotlib}
\novonome{\numpy}{NumPy}
\novonome{\scipy}{SciPy}
\novonome{\pandas}{Pandas}

% pacotes extras
\usepackage{caption, subcaption, pdfpages, float}
\usepackage{circuitikz, graphics, wrapfig}

\usepackage{caption}
\usepackage[newfloat]{minted}
\definecolor{sepia}{RGB}{252,246,226}
\setminted{
    autogobble,
    bgcolor=sepia,
    style=pastie,
    python3
}
\setmintedinline{
    bgcolor={}
}

\newenvironment{code}{
    \captionsetup{type=listing}
}{}
\SetupFloatingEnvironment{listing}{name=Código}

\newcommand{\pyinput}[4][]{
    \begin{code}
        \captionof{listing}{#3}
        \label{#4}
        \inputminted[firstline=3,#1]{python}{recursos/#2.py}
    \end{code}
}
\newcommand{\pyline}[1]{\mintinline{python}{#1}}
\newcommand{\novopynome}[2]{
    \newcommand{#1}{%
        \pyline{#2}
    }
}
\novopynome{\dataframe}{DataFrame}

% começa a seção no `0`
\setcounter{section}{-1}
